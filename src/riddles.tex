\documentclass{article}
%\usepackage[pdftex]{} % for pdf features
\usepackage{thumbpdf} % for thumbnails
%\usepacakge{tex4ht}
\usepackage[pdftex,
	pdftitle={Riddles},
	pdfauthor={Mark Veltzer},
	pdfsubject={A collection of riddles},
	pdfkeywords={riddles, riddling, math, mathematics, geometry, combinatorics, Mark Veltzer, veltzer, Veltzer},
	pdfproducer={Mark Veltzer <mark.veltzer@gmail.com>},
	]{hyperref}
\pdfminorversion=5
\pdfoutput=1
\pdfcompresslevel=0
%\usepackage{pdftricks}
%\usepackage{pst-all}
% if you use the following package straight latex running (not via pdflatex), will not work...
\usepackage{pst-pdf}
%\usepackage{auto-pst-pdf}
%	colorlinks=true,
%	urlcolor=rltblue, % \href{...}{...} external (URL)
%	filecolor=rltgreen, % \href{...} local file
%	linkcolor=rltred, % \ref{...} and \pageref{...}
%	pagebackref,
%	pdfpagemode=UseNone,
%	bookmarksopen=true]{hyperref} % needed for sketch
%\definecolor{rltred}{rgb}{0.75,0,0}
%\definecolor{rltgreen}{rgb}{0,0.5,0}
%\definecolor{rltblue}{rgb}{0,0,0.75}
%\usepackage{amsmath,color,thumbpdf,html,hyperref}
%\usepackage{pst-barcode,pstricks-add}

% these are the packages that we actually use
%\usepackage{hyperref} % for hyperlinks
%\usepackage{color} % for colors (creates errors when applied with pdftex option)
\usepackage{amsmath} % for math (does not pdftex option)
\usepackage{amssymb} % for math symbols (does not pdftex option)
%\usepackage{pstricks} % for sketch with pstricks support (default for sketch)
\usepackage{tikz} % for sketch with tikz (does not pdftex option)

\title{Riddles}
\author{Mark Veltzer}
\date{\today}

\begin{document}

\maketitle

\tableofcontents

\section{Rational points on a circle (Calculus)}

%\subsection{Question}

Could there be a circle on the plane where only one point has rational co-ordinates? A point $p=(x,y)$ has rational co-ordinates iff $x,y\in\mathbb{Q}$. What about a sphere? Are there circles with more than one rational point? Are there circles with infinite rational points? Can you find a circle with exactly $n$ rational points on it for each $n\in\mathbb{N}$?

\subsection{Solution}

Yes, there could. Consider the circle: $(x-r)^2+y^2=r^2$ for some irrational $r$.
$(0,0)$ is a point on this circle and is rational. But from the equation it follows
that: $x^2-2xr+r^2+y^2=r^2$ or $x^2+y^2=2xr$ or $r=(x^2+y^2)/2x$. If there
was a rational solution to this equation where $x\ne0$ then it would follow that
$r$ is rational since it is the result of multiplication, division, addition and subtration
of rational numbers and this is a contradiction. This means this equation can only
have rational solutions for $x=0$. If $x=0$ then $r^2+y^2=r^2$ and so $y=0$ and so
$(0,0)$ is the only such solution.

Same solution applies to a sphere: $(x-r)^2+y^2+z^2=r^2$ for some irrational $r$.
In this case: $r=(x^2+y^2+z^2)/2x$. Which means that for $x\ne0$ any rational soultion would imply a rational $r$.
$(0,0,0)$ is therefor the only rational solution.

Yes, there are circles with more than one rational point. Take $x^2+y^2=2$ which is the circle whose
center is at $(0,0)$ and whose radius is $\sqrt{2}$. The points $(1,1),(-1,1),(1,-1),(-1,-1)$ are four rational
points which are on this circle.

Yes, there are circles with infinite rational points (TODO).

What about for each $n\in\mathbb{N}$? (TODO).

\section{Always lead in election (Combinatorics)}

%\subsection{Question}

%id=elections (for out of source references to this riddle)

An election was voted a perfect tie (even number of electors) and was decided by way of putting
all the votes into a hat, mixing them uniformly and counting them one by one. What is the chance
that one of the candidates was leading during the entire counting process?

\subsection{Solution}

The question could be rephrased as how many graphs that go either one step up or one step down (lets call those binary graphs) do not go down below zero out of the set of all graphs that go from zero to zero. At first lets note that in the following solution $n$ is even and so there is no problem with $n/2$. A first observation is that the set of all graphs going from 0 to 0 in $n$ steps is ${n \choose n/2}$.

Let's try to count the graphs that tip below 0. Every graph that tips below 0 will reach -1 at one or more points. Lets find the first place where the graph reaches -1. From that point on the graph climbs one more than it descends since it finally reaches 0. Lets flip it from that point on - meaning switch every climb with descent and every descent with a climb. Now the graph descends one more than it climbs and since it starts from that point at height -1 then it now reaches a height -2 at $n$ instead of the original 0.

The main proposition is that \emph{every} graph which travels from 0 to -2 correspons with a \emph{1-1 correspondance} to graphs that travel from 0 to 0 and reach -1 at some point (convince yourself
of this).

If this is so then the number of graphs that go from 0 to -2 is ${n \choose n/2+1}$ since we need to
choose points at which the graph will go down and we have $n/2+1$ descends. If so then the number of
graphs that do not reach -1 is:

${n \choose n/2} - {n \choose n/2+1} = \frac{(n)!}{(n/2+1)!(n/2)!} = \frac{1}{n/2+1}{n \choose n/2}$.

And now back to the original question "What is the chance that one of the candidates was leading during the entire counting process?". Lets divide the latter result with the former and get:

$\frac{1}{n/2+1}{n \choose n/2} / {n \choose n/2} = \frac{1}{n/2+1}$

The result is all about \htmladdnormallink{Catalan numbers}{http://en.wikipedia.org/wiki/Catalan_number}
and \htmladdnormallink{Bertrand's ballot theorem}{http://en.wikipedia.org/wiki/Bertrand\%27s_ballot_theorem}.

One of the interesting results of this question is that ${n \choose n/2}$ is divisible by $n/2+1$. One reason
is the argument stated above but a more direct number theory based argument may be found. A more general result
relating to ${n \choose k}$ for any $k$ may also be obtained (TODO).

\section{Four Coke bottles on a table (Geometry)}

Arrange four Coke bottles on a table so that the distances between each pair of caps will be the same. Assume that
you can make a Coke bottle stand on it's head (this is a pretty strong hint!, maybe I should remove it ?!?). The length
or height of each bottle is ${h}$.

\subsection{Solution}

There are two solutions:

The first solution is to put the bottles in a square where the side of the square is the length of the bottle. Bottles
next to each other will stand \emph{in reverse} (if the first is on it's head the the second will stand straight).

\begin{center}
\input{out/bottles.tex}
\end{center}

The shape created by the bottle caps is, ofcourse, a perfect triangular pyramid with a side of length $d=\sqrt{2h^2}$.

The second solution is to put three bottles on the vertices of an equilateral triangle standing one way. The fourth will be placed
in the middle of the triangle and will be standing reversed to the others. The problem here is to calculate the side of the triangle.
Lets use the following diagram:

\begin{center}
\input{out/perfect_triangular_pyramid.tex}
\end{center}

What we are trying to find is $d$ as a function of $h$. Since $H$, $d/2$ and $d$ create a right angeled triangle, it follows
thae $H^2+(d/2)^2=d^2$ and so $H=\sqrt{3}/2d$. Since the size of the inner triangle could be computed two ways then $hH=dH'$ and
since $H'=\sqrt{H^2-(d/2)^2}$ it follows after some math that $d=\sqrt{3/2}h$ which makes $d$ about $1.224h$.

The solution built on this analysis will look like this:

\begin{center}
\input{out/bottles2.tex}
\end{center}

As is evident the key to solving this question is the perfect triangular pyramid as both solutions are based on it and it is the only
way to put 4 elements in space being spaced equally apart. In theory you could place the bottles at any height and so
infinite solutions to this riddle could be produced by making the side of the perfect pyramid as small or as large as you want. But we are
constrained by the table and so the heights at which the caps are will be either $0$ or $h$. The two solutions above seem to be primary
examples under this constraint. Minor variations on the above two solutions could be produced by making the bottles whose caps are at table
level lay instead of stand on their heads. The first solution is more attractive than the second because it could be performed in practice
without relying on measurement tools in order to place the bottles at the calculated distance from each other as per the second solution.

\section{Strange Trip (Geometry)}

You travel 100 miles north, 100 miles east, and then 100 miles south. You are at the same point that you started from. Describe all the places on earth this could be, if any.

\subsection{Solution}

At the south pole. Or, 100 miles south of any point on the circle around the north pole that is 100 miles in circumference. Or, 100 miles south of any such circle whose circumference is an integral fraction of 100 miles.

\section{Three Cards (Probability)}

There are 3 cards: one is all red, one is all blue, and the third is blue on one side and red on the other. The cards are shuffled. You pick one at random and it is blue on the side facing you. What are the chances that it is also blue on the other side?

\subsection{Solution}

2/3. 2 of the 3 blue sides have blue on the other side. You might think the odds are 50/50 because you could have one of two cards, but you actually get a side at random, not a card, so you are more likely to have the blue/blue card than the red/blue card.

Here is another way to think of it: When you pick a card at random, the chances of it being the same on both sides (either all red or all blue) are 2 out of 3, and this doesn't change just because you see one of the sides. If they did, they'd change no matter which color you saw, and it wouldn't even matter if you looked. So when you do see the blue side, the red/red card is eliminated, but the blue/blue card still has a 2/3 probability.

\section{Monty Hall Problem (Probability)}

Behind 1 of 3 closed doors is a prize. You pick one of the doors. Monty opens one of the other doors and it is empty. You are given the choice of sticking with your choice or switching to the other unopened door. Should you switch? What are your chances of winning if you do?

\subsection{Solution}

You should switch. Your chances are 2/3 if you do. The initial chances of your first choice were 1/3, and opening another door without the prize doesn't actually change that, so the remaining door now has a 2/3 chance because the chances of all possibilities must sum to 1.

Many people think the chances are 50/50 because there are two doors left, but that is not correct. If you're not convinced 2/3 is correct, consider a 100 door version: you pick 1 door and Monty opens 98 other doors avoiding the one with the prize. Now you have a 99\% chance if you switch.

\label{end}\end{document}
